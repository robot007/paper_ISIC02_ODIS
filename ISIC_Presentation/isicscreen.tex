\documentclass[a4paper,12pt]{article}
\usepackage{xspace,colortbl,layout} 

%\usepackage[screen,panelleft,gray,paneltoc]{pdfscreen}
\usepackage[screen,gray]{pdfscreen}
\margins{1cm}{1cm}{1cm}{1cm}                     
\screensize{6.25in}{8in}

    \definecolor{section6}{rgb}{1,.1,.1}
    \definecolor{panelbackground}{rgb}{0.1,0.1,0.5}
\panelwidth=1.3in
\def\panel{%\colorbox{panelbackground}
{\begin{minipage}[t][\paperheight][b]{\panelwidth}
    \centering\null\vspace*{12pt}
    \includegraphics[width=.75in]{Sample}\par\vfill
%    \href{\@urlid}{\addButton{.85in}{Homepage}}\par\vfill
\Acrobatmenu{FirstPage}{\addButton{.85in}
    {\FBlack First Page}}\par\vfill %
    %
\Acrobatmenu{FirstPage}{\addButton{.2in}
    {\FBlack\scalebox{.8}[1.4]{\btl\btl}}}\hspace{-3pt}
\Acrobatmenu{PrevPage}{\addButton{.2in}
    {\FBlack\scalebox{.8}[1.4]{\btl}}}\hspace{-3pt}
\Acrobatmenu{NextPage}{\addButton{.2in}
    {\LBlack\scalebox{.8}[1.4]{\rtl}}}\hspace{-3pt}
\Acrobatmenu{LastPage}{\addButton{.2in}
    {\LBlack\scalebox{.8}[1.4]{\rtl\rtl}}}\par\vfill %
    %
\Acrobatmenu{GoBack}{\addButton{.85in}
    {Go Back}}\par\vfill
\Acrobatmenu{FullScreen}{\addButton{.85in}{Full Screen}}\par\vfill
\Acrobatmenu{Close}{\addButton{.85in}{Close}}\par\vfill
\Acrobatmenu{Quit}{\addButton{.85in}{Quit}}\par
    \null\vspace*{12pt}
\end{minipage}}}

%\overlay{lightsteelblue.pdf}
\paneloverlay{Sample}

\renewcommand{\baselinestretch}{1.0} 


\begin{document}
\layout
\bf \Large

\begin{screen}
    \title{\color{section1}\Huge Some Sensing and Perception Techniques for an
Omnidirectional Ground Vehicle with a Laser Scanner}

\author{\color{black}{Zhen Song, YangQuan Chen, Lili Ma and You Chung Chung}}
\maketitle
\begin{abstract}
 This paper presents some techniques for sensing and perception for an
omnidirectional ground autonomous vehicle equipped with a laser scanner. In an
assumed structured environment, the sensor data processing methods for both 1D
and 2D laser scanners, are discussed. Raw data are segmented to lines, circles,
ellipses, planes and corners by task dependent segmentation algorithms.  Each
subset of data is then fit by known template shapes as listed above.  An arbitrator is used to discriminate the shape for the final object identification.

\vspace{1cm}
{\bf Key Words: } 
 Hough transform;
 line fitting;
 corner fitting;
 arc/circle fitting;
 ellipse fitting;
 algebraic fitting;
 object arbitrator;
 Omni-directional vehicle (ODV).
\end{abstract}
\end{screen}

%\begin{slide} %2
%\small
%\noindent abstract:
%
% This paper presents some techniques for sensing and perception for an
%omnidirectional ground autonomous vehicle equipped with a laser scanner. In an
%assumed structured environment, the sensor data processing methods for both 1D
%and 2D laser scanners, are discussed. Raw data are segmented to lines, circles,
%ellipses, planes and corners by task dependent segmentation algorithms.  Each
%subset of data is then fit by known template shapes as listed above.  An arbitrator is used to discriminate the shape for the final object identification.
%
%\vspace{1cm}
%{\bf Key Words: } 
% Hough transform;
% line fitting;
% corner fitting;
% arc/circle fitting;
% ellipse fitting;
% algebraic fitting;
% object arbitrator;
% Omni-directional vehicle (ODV).
%\end{slide}


\begin{slide} %3
\centerline{\Huge Title }\vspace{0.5cm}
    \begin{itemize}
    \item 
    \end{itemize}
        
\end{slide}


\begin{slide}
\centerline{\Huge Adjust  }\vspace{0.5cm}
    \begin{itemize}
    \item 123456789012345678901234567890123456789012345678901234567890123456789012345678901234567890123456789012345678901234567890123456789012345678901234567890123456789012345678901234567890123456789012345678901234567890123456789012345678901234567890123456789012345678901234567890123456789012345678901234567890
    \item 123456789012345678901234567890123456789012345678901234567890123456789012345678901234567890123456789012345678901234567890123456789012345678901234567890123456789012345678901234567890123456789012345678901234567890123456789012345678901234567890123456789012345678901234567890123456789012345678901234567890
    \item 123456789012345678901234567890123456789012345678901234567890123456789012345678901234567890123456789012345678901234567890123456789012345678901234567890123456789012345678901234567890123456789012345678901234567890123456789012345678901234567890123456789012345678901234567890123456789012345678901234567890
    \item 123456789012345678901234567890123456789012345678901234567890123456789012345678901234567890123456789012345678901234567890123456789012345678901234567890123456789012345678901234567890123456789012345678901234567890123456789012345678901234567890123456789012345678901234567890123456789012345678901234567890
    \item 123456789012345678901234567890123456789012345678901234567890123456789012345678901234567890123456789012345678901234567890123456789012345678901234567890123456789012345678901234567890123456789012345678901234567890123456789012345678901234567890123456789012345678901234567890123456789012345678901234567890
    \item 123456789012345678901234567890123456789012345678901234567890123456789012345678901234567890123456789012345678901234567890123456789012345678901234567890123456789012345678901234567890123456789012345678901234567890123456789012345678901234567890123456789012345678901234567890123456789012345678901234567890
    \item 123456789012345678901234567890123456789012345678901234567890123456789012345678901234567890123456789012345678901234567890123456789012345678901234567890123456789012345678901234567890123456789012345678901234567890123456789012345678901234567890123456789012345678901234567890123456789012345678901234567890                
    \item 123456789012345678901234567890123456789012345678901234567890123456789012345678901234567890123456789012345678901234567890123456789012345678901234567890123456789012345678901234567890123456789012345678901234567890123456789012345678901234567890123456789012345678901234567890123456789012345678901234567890
    \item 123456789012345678901234567890123456789012345678901234567890123456789012345678901234567890123456789012345678901234567890123456789012345678901234567890123456789012345678901234567890123456789012345678901234567890123456789012345678901234567890123456789012345678901234567890123456789012345678901234567890
    \item 123456789012345678901234567890123456789012345678901234567890123456789012345678901234567890123456789012345678901234567890123456789012345678901234567890123456789012345678901234567890123456789012345678901234567890123456789012345678901234567890123456789012345678901234567890123456789012345678901234567890
    \end{itemize}
    
\end{slide}


\end{document}

